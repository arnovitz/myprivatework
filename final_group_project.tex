% Options for packages loaded elsewhere
\PassOptionsToPackage{unicode}{hyperref}
\PassOptionsToPackage{hyphens}{url}
%
\documentclass[
]{article}
\usepackage{amsmath,amssymb}
\usepackage{lmodern}
\usepackage{iftex}
\ifPDFTeX
  \usepackage[T1]{fontenc}
  \usepackage[utf8]{inputenc}
  \usepackage{textcomp} % provide euro and other symbols
\else % if luatex or xetex
  \usepackage{unicode-math}
  \defaultfontfeatures{Scale=MatchLowercase}
  \defaultfontfeatures[\rmfamily]{Ligatures=TeX,Scale=1}
\fi
% Use upquote if available, for straight quotes in verbatim environments
\IfFileExists{upquote.sty}{\usepackage{upquote}}{}
\IfFileExists{microtype.sty}{% use microtype if available
  \usepackage[]{microtype}
  \UseMicrotypeSet[protrusion]{basicmath} % disable protrusion for tt fonts
}{}
\makeatletter
\@ifundefined{KOMAClassName}{% if non-KOMA class
  \IfFileExists{parskip.sty}{%
    \usepackage{parskip}
  }{% else
    \setlength{\parindent}{0pt}
    \setlength{\parskip}{6pt plus 2pt minus 1pt}}
}{% if KOMA class
  \KOMAoptions{parskip=half}}
\makeatother
\usepackage{xcolor}
\usepackage[margin=1in]{geometry}
\usepackage{color}
\usepackage{fancyvrb}
\newcommand{\VerbBar}{|}
\newcommand{\VERB}{\Verb[commandchars=\\\{\}]}
\DefineVerbatimEnvironment{Highlighting}{Verbatim}{commandchars=\\\{\}}
% Add ',fontsize=\small' for more characters per line
\usepackage{framed}
\definecolor{shadecolor}{RGB}{248,248,248}
\newenvironment{Shaded}{\begin{snugshade}}{\end{snugshade}}
\newcommand{\AlertTok}[1]{\textcolor[rgb]{0.94,0.16,0.16}{#1}}
\newcommand{\AnnotationTok}[1]{\textcolor[rgb]{0.56,0.35,0.01}{\textbf{\textit{#1}}}}
\newcommand{\AttributeTok}[1]{\textcolor[rgb]{0.77,0.63,0.00}{#1}}
\newcommand{\BaseNTok}[1]{\textcolor[rgb]{0.00,0.00,0.81}{#1}}
\newcommand{\BuiltInTok}[1]{#1}
\newcommand{\CharTok}[1]{\textcolor[rgb]{0.31,0.60,0.02}{#1}}
\newcommand{\CommentTok}[1]{\textcolor[rgb]{0.56,0.35,0.01}{\textit{#1}}}
\newcommand{\CommentVarTok}[1]{\textcolor[rgb]{0.56,0.35,0.01}{\textbf{\textit{#1}}}}
\newcommand{\ConstantTok}[1]{\textcolor[rgb]{0.00,0.00,0.00}{#1}}
\newcommand{\ControlFlowTok}[1]{\textcolor[rgb]{0.13,0.29,0.53}{\textbf{#1}}}
\newcommand{\DataTypeTok}[1]{\textcolor[rgb]{0.13,0.29,0.53}{#1}}
\newcommand{\DecValTok}[1]{\textcolor[rgb]{0.00,0.00,0.81}{#1}}
\newcommand{\DocumentationTok}[1]{\textcolor[rgb]{0.56,0.35,0.01}{\textbf{\textit{#1}}}}
\newcommand{\ErrorTok}[1]{\textcolor[rgb]{0.64,0.00,0.00}{\textbf{#1}}}
\newcommand{\ExtensionTok}[1]{#1}
\newcommand{\FloatTok}[1]{\textcolor[rgb]{0.00,0.00,0.81}{#1}}
\newcommand{\FunctionTok}[1]{\textcolor[rgb]{0.00,0.00,0.00}{#1}}
\newcommand{\ImportTok}[1]{#1}
\newcommand{\InformationTok}[1]{\textcolor[rgb]{0.56,0.35,0.01}{\textbf{\textit{#1}}}}
\newcommand{\KeywordTok}[1]{\textcolor[rgb]{0.13,0.29,0.53}{\textbf{#1}}}
\newcommand{\NormalTok}[1]{#1}
\newcommand{\OperatorTok}[1]{\textcolor[rgb]{0.81,0.36,0.00}{\textbf{#1}}}
\newcommand{\OtherTok}[1]{\textcolor[rgb]{0.56,0.35,0.01}{#1}}
\newcommand{\PreprocessorTok}[1]{\textcolor[rgb]{0.56,0.35,0.01}{\textit{#1}}}
\newcommand{\RegionMarkerTok}[1]{#1}
\newcommand{\SpecialCharTok}[1]{\textcolor[rgb]{0.00,0.00,0.00}{#1}}
\newcommand{\SpecialStringTok}[1]{\textcolor[rgb]{0.31,0.60,0.02}{#1}}
\newcommand{\StringTok}[1]{\textcolor[rgb]{0.31,0.60,0.02}{#1}}
\newcommand{\VariableTok}[1]{\textcolor[rgb]{0.00,0.00,0.00}{#1}}
\newcommand{\VerbatimStringTok}[1]{\textcolor[rgb]{0.31,0.60,0.02}{#1}}
\newcommand{\WarningTok}[1]{\textcolor[rgb]{0.56,0.35,0.01}{\textbf{\textit{#1}}}}
\usepackage{longtable,booktabs,array}
\usepackage{calc} % for calculating minipage widths
% Correct order of tables after \paragraph or \subparagraph
\usepackage{etoolbox}
\makeatletter
\patchcmd\longtable{\par}{\if@noskipsec\mbox{}\fi\par}{}{}
\makeatother
% Allow footnotes in longtable head/foot
\IfFileExists{footnotehyper.sty}{\usepackage{footnotehyper}}{\usepackage{footnote}}
\makesavenoteenv{longtable}
\usepackage{graphicx}
\makeatletter
\def\maxwidth{\ifdim\Gin@nat@width>\linewidth\linewidth\else\Gin@nat@width\fi}
\def\maxheight{\ifdim\Gin@nat@height>\textheight\textheight\else\Gin@nat@height\fi}
\makeatother
% Scale images if necessary, so that they will not overflow the page
% margins by default, and it is still possible to overwrite the defaults
% using explicit options in \includegraphics[width, height, ...]{}
\setkeys{Gin}{width=\maxwidth,height=\maxheight,keepaspectratio}
% Set default figure placement to htbp
\makeatletter
\def\fps@figure{htbp}
\makeatother
\setlength{\emergencystretch}{3em} % prevent overfull lines
\providecommand{\tightlist}{%
  \setlength{\itemsep}{0pt}\setlength{\parskip}{0pt}}
\setcounter{secnumdepth}{-\maxdimen} % remove section numbering
\ifLuaTeX
  \usepackage{selnolig}  % disable illegal ligatures
\fi
\IfFileExists{bookmark.sty}{\usepackage{bookmark}}{\usepackage{hyperref}}
\IfFileExists{xurl.sty}{\usepackage{xurl}}{} % add URL line breaks if available
\urlstyle{same} % disable monospaced font for URLs
\hypersetup{
  pdftitle={Final Group 18 project},
  pdfauthor={Group 18 Paula Lagos, Ignacio Nino, Nicholas Arnovitz, Isabel Gonzalez},
  hidelinks,
  pdfcreator={LaTeX via pandoc}}

\title{Final Group 18 project}
\author{Group 18 Paula Lagos, Ignacio Nino, Nicholas Arnovitz, Isabel
Gonzalez}
\date{2023-06-08}

\begin{document}
\maketitle

{
\setcounter{tocdepth}{2}
\tableofcontents
}
\hypertarget{the-problem-predicting-credit-card-fraud}{%
\section{The problem: predicting credit card
fraud}\label{the-problem-predicting-credit-card-fraud}}

The goal of the project is to predict fraudulent credit card
transactions.

We will be using a dataset with credit card transactions containing
legitimate and fraud transactions. Fraud is typically well below 1\% of
all transactions, so a naive model that predicts that all transactions
are legitimate and not fraudulent would have an accuracy of well over
99\%-- pretty good, no?

You can read more on credit card fraud on
\href{https://www.scirp.org/journal/paperinformation.aspx?paperid=105944}{Credit
Card Fraud Detection Using Weighted Support Vector Machine}

The dataset we will use consists of credit card transactions and it
includes information about each transaction including customer details,
the merchant and category of purchase, and whether or not the
transaction was a fraud.

\hypertarget{obtain-the-data}{%
\subsection{Obtain the data}\label{obtain-the-data}}

The dataset is too large to be hosted on Canvas or Github, so please
download it from dropbox
\url{https://www.dropbox.com/sh/q1yk8mmnbbrzavl/AAAxzRtIhag9Nc_hODafGV2ka?dl=0}
and save it in your \texttt{dsb} repo, under the \texttt{data} folder.

As we will be building a classifier model using tidymodels, there's two
things we need to do:

\begin{enumerate}
\def\labelenumi{\arabic{enumi}.}
\tightlist
\item
  Define the outcome variable \texttt{is\_fraud} as a factor, or
  categorical, variable, instead of the numerical 0-1 varaibles.
\item
  In tidymodels, the first level is the event of interest. If we leave
  our data as is, \texttt{0} is the first level, but we want to find out
  when we actually did (\texttt{1}) have a fraudulent transaction
\end{enumerate}

\begin{verbatim}
## Rows: 671,028
## Columns: 14
## $ trans_date_trans_time <dttm> 2019-02-22 07:32:58, 2019-02-16 15:07:20, 2019-~
## $ trans_year            <dbl> 2019, 2019, 2019, 2019, 2019, 2019, 2019, 2020, ~
## $ category              <chr> "entertainment", "kids_pets", "personal_care", "~
## $ amt                   <dbl> 7.79, 3.89, 8.43, 40.00, 54.04, 95.61, 64.95, 3.~
## $ city                  <chr> "Veedersburg", "Holloway", "Arnold", "Apison", "~
## $ state                 <chr> "IN", "OH", "MO", "TN", "CO", "GA", "MN", "AL", ~
## $ lat                   <dbl> 40.1186, 40.0113, 38.4305, 35.0149, 39.4584, 32.~
## $ long                  <dbl> -87.2602, -80.9701, -90.3870, -85.0164, -106.385~
## $ city_pop              <dbl> 4049, 128, 35439, 3730, 277, 1841, 136, 190178, ~
## $ job                   <chr> "Development worker, community", "Child psychoth~
## $ dob                   <date> 1959-10-19, 1946-04-03, 1985-03-31, 1991-01-28,~
## $ merch_lat             <dbl> 39.41679, 39.74585, 37.73078, 34.53277, 39.95244~
## $ merch_long            <dbl> -87.52619, -81.52477, -91.36875, -84.10676, -106~
## $ is_fraud              <fct> 0, 0, 0, 0, 0, 0, 0, 0, 0, 0, 0, 0, 0, 0, 0, 0, ~
\end{verbatim}

The data dictionary is as follows

\begin{longtable}[]{@{}ll@{}}
\toprule()
column(variable) & description \\
\midrule()
\endhead
trans\_date\_trans\_time & Transaction DateTime \\
trans\_year & Transaction year \\
category & category of merchant \\
amt & amount of transaction \\
city & City of card holder \\
state & State of card holder \\
lat & Latitude location of purchase \\
long & Longitude location of purchase \\
city\_pop & card holder's city population \\
job & job of card holder \\
dob & date of birth of card holder \\
merch\_lat & Latitude Location of Merchant \\
merch\_long & Longitude Location of Merchant \\
is\_fraud & Whether Transaction is Fraud (1) or Not (0) \\
\bottomrule()
\end{longtable}

We also add some of the variables we considered in our EDA for this
dataset during homework 2.

\begin{Shaded}
\begin{Highlighting}[]
\NormalTok{card\_fraud }\OtherTok{\textless{}{-}}\NormalTok{ card\_fraud }\SpecialCharTok{\%\textgreater{}\%} 
  \FunctionTok{mutate}\NormalTok{( }\AttributeTok{hour =} \FunctionTok{hour}\NormalTok{(trans\_date\_trans\_time),}
          \AttributeTok{wday =} \FunctionTok{wday}\NormalTok{(trans\_date\_trans\_time, }\AttributeTok{label =} \ConstantTok{TRUE}\NormalTok{),}
          \AttributeTok{month\_name =} \FunctionTok{month}\NormalTok{(trans\_date\_trans\_time, }\AttributeTok{label =} \ConstantTok{TRUE}\NormalTok{),}
          \AttributeTok{age =} \FunctionTok{interval}\NormalTok{(dob, trans\_date\_trans\_time) }\SpecialCharTok{/} \FunctionTok{years}\NormalTok{(}\DecValTok{1}\NormalTok{)}
\NormalTok{) }\SpecialCharTok{\%\textgreater{}\%} 
  \FunctionTok{rename}\NormalTok{(}\AttributeTok{year =}\NormalTok{ trans\_year) }\SpecialCharTok{\%\textgreater{}\%} 
  
  \FunctionTok{mutate}\NormalTok{(}
    
    \CommentTok{\# convert latitude/longitude to radians}
    \AttributeTok{lat1\_radians =}\NormalTok{ lat }\SpecialCharTok{/} \FloatTok{57.29577951}\NormalTok{,}
    \AttributeTok{lat2\_radians =}\NormalTok{ merch\_lat }\SpecialCharTok{/} \FloatTok{57.29577951}\NormalTok{,}
    \AttributeTok{long1\_radians =}\NormalTok{ long }\SpecialCharTok{/} \FloatTok{57.29577951}\NormalTok{,}
    \AttributeTok{long2\_radians =}\NormalTok{ merch\_long }\SpecialCharTok{/} \FloatTok{57.29577951}\NormalTok{,}
    
    \CommentTok{\# calculate distance in miles}
    \AttributeTok{distance\_miles =} \FloatTok{3963.0} \SpecialCharTok{*} \FunctionTok{acos}\NormalTok{((}\FunctionTok{sin}\NormalTok{(lat1\_radians) }\SpecialCharTok{*} \FunctionTok{sin}\NormalTok{(lat2\_radians)) }\SpecialCharTok{+} \FunctionTok{cos}\NormalTok{(lat1\_radians) }\SpecialCharTok{*} \FunctionTok{cos}\NormalTok{(lat2\_radians) }\SpecialCharTok{*} \FunctionTok{cos}\NormalTok{(long2\_radians }\SpecialCharTok{{-}}\NormalTok{ long1\_radians)),}

    \CommentTok{\# calculate distance in km}
    \AttributeTok{distance\_km =} \FloatTok{6377.830272} \SpecialCharTok{*} \FunctionTok{acos}\NormalTok{((}\FunctionTok{sin}\NormalTok{(lat1\_radians) }\SpecialCharTok{*} \FunctionTok{sin}\NormalTok{(lat2\_radians)) }\SpecialCharTok{+} \FunctionTok{cos}\NormalTok{(lat1\_radians) }\SpecialCharTok{*} \FunctionTok{cos}\NormalTok{(lat2\_radians) }\SpecialCharTok{*} \FunctionTok{cos}\NormalTok{(long2\_radians }\SpecialCharTok{{-}}\NormalTok{ long1\_radians))}

\NormalTok{  )}
\end{Highlighting}
\end{Shaded}

\hypertarget{exploratory-data-analysis-eda}{%
\subsection{Exploratory Data Analysis
(EDA)}\label{exploratory-data-analysis-eda}}

You have done some EDA and you can pool together your group's expertise
in which variables to use as features. You can reuse your EDA from
earlier, but we expect at least a few visualisations and/or tables to
explore the dataset and identify any useful features.

Group all variables by type and examine each variable class by class.
The dataset has the following types of variables:

\begin{enumerate}
\def\labelenumi{\arabic{enumi}.}
\tightlist
\item
  Strings
\item
  Geospatial Data
\item
  Dates
\item
  Date/Times
\item
  Numerical
\end{enumerate}

Strings are usually not a useful format for classification problems. The
strings should be converted to factors, dropped, or otherwise
transformed.

\textbf{\emph{Strings to Factors}}

\begin{itemize}
\tightlist
\item
  \texttt{category}, Category of Merchant
\item
  \texttt{job}, Job of Credit Card Holder
\end{itemize}

\textbf{\emph{Strings to Geospatial Data}}

We have plenty of geospatial data as lat/long pairs, so I want to
convert city/state to lat/long so I can compare to the other geospatial
variables. This will also make it easier to compute new variables like
the distance the transaction is from the home location.

\begin{itemize}
\tightlist
\item
  \texttt{city}, City of Credit Card Holder
\item
  \texttt{state}, State of Credit Card Holder
\end{itemize}

\hypertarget{exploring-factors-how-is-the-compactness-of-categories}{%
\subsection{Exploring factors: how is the compactness of
categories?}\label{exploring-factors-how-is-the-compactness-of-categories}}

\begin{itemize}
\tightlist
\item
  Do we have excessive number of categories? Do we want to combine some?
\end{itemize}

\begin{Shaded}
\begin{Highlighting}[]
\NormalTok{card\_fraud }\SpecialCharTok{\%\textgreater{}\%} 
  \FunctionTok{count}\NormalTok{(category, }\AttributeTok{sort=}\ConstantTok{TRUE}\NormalTok{)}\SpecialCharTok{\%\textgreater{}\%} 
  \FunctionTok{mutate}\NormalTok{(}\AttributeTok{perc =}\NormalTok{ n}\SpecialCharTok{/}\FunctionTok{sum}\NormalTok{(n))}
\end{Highlighting}
\end{Shaded}

\begin{verbatim}
## # A tibble: 14 x 3
##    category           n   perc
##    <chr>          <int>  <dbl>
##  1 gas_transport  68046 0.101 
##  2 grocery_pos    63791 0.0951
##  3 home           63597 0.0948
##  4 shopping_pos   60416 0.0900
##  5 kids_pets      58772 0.0876
##  6 shopping_net   50743 0.0756
##  7 entertainment  48521 0.0723
##  8 food_dining    47527 0.0708
##  9 personal_care  46843 0.0698
## 10 health_fitness 44341 0.0661
## 11 misc_pos       41244 0.0615
## 12 misc_net       32829 0.0489
## 13 grocery_net    23485 0.0350
## 14 travel         20873 0.0311
\end{verbatim}

\begin{Shaded}
\begin{Highlighting}[]
\NormalTok{card\_fraud }\SpecialCharTok{\%\textgreater{}\%} 
  \FunctionTok{count}\NormalTok{(job, }\AttributeTok{sort=}\ConstantTok{TRUE}\NormalTok{) }\SpecialCharTok{\%\textgreater{}\%} 
  \FunctionTok{mutate}\NormalTok{(}\AttributeTok{perc =}\NormalTok{ n}\SpecialCharTok{/}\FunctionTok{sum}\NormalTok{(n))}
\end{Highlighting}
\end{Shaded}

\begin{verbatim}
## # A tibble: 494 x 3
##    job                            n    perc
##    <chr>                      <int>   <dbl>
##  1 Film/video editor           5106 0.00761
##  2 Exhibition designer         4728 0.00705
##  3 Naval architect             4546 0.00677
##  4 Surveyor, land/geomatics    4448 0.00663
##  5 Materials engineer          4292 0.00640
##  6 Designer, ceramics/pottery  4262 0.00635
##  7 IT trainer                  4014 0.00598
##  8 Financial adviser           3959 0.00590
##  9 Systems developer           3948 0.00588
## 10 Environmental consultant    3831 0.00571
## # i 484 more rows
\end{verbatim}

The predictors \texttt{category} and \texttt{job} are transformed into
factors.

\begin{Shaded}
\begin{Highlighting}[]
\NormalTok{card\_fraud }\OtherTok{\textless{}{-}}\NormalTok{ card\_fraud }\SpecialCharTok{\%\textgreater{}\%} 
  \FunctionTok{mutate}\NormalTok{(}\AttributeTok{category =} \FunctionTok{factor}\NormalTok{(category),}
         \AttributeTok{job =} \FunctionTok{factor}\NormalTok{(job))}
\end{Highlighting}
\end{Shaded}

\texttt{category} has 14 unique values, and \texttt{job} has 494 unique
values. The dataset is quite large, with over 670K records, so these
variables don't have an excessive number of levels at first glance.
However, it is worth seeing if we can compact the levels to a smaller
number.

\hypertarget{why-do-we-care-about-the-number-of-categories-and-whether-they-are-excessive}{%
\subsubsection{Why do we care about the number of categories and whether
they are
``excessive''?}\label{why-do-we-care-about-the-number-of-categories-and-whether-they-are-excessive}}

Consider the extreme case where a dataset had categories that only
contained one record each. There is simply insufficient data to make
correct predictions using category as a predictor on new data with that
category label. Additionally, if your modeling uses dummy variables,
having an extremely large number of categories will lead to the
production of a huge number of predictors, which can slow down the
fitting. This is fine if all the predictors are useful, but if they
aren't useful (as in the case of having only one record for a category),
trimming them will improve the speed and quality of the data fitting.

If I had subject matter expertise, I could manually combine categories.
If you don't have subject matter expertise, or if performing this task
would be too labor intensive, then you can use cutoffs based on the
amount of data in a category. If the majority of the data exists in only
a few categories, then it might be reasonable to keep those categories
and lump everything else in an ``other'' category or perhaps even drop
the data points in smaller categories.

\hypertarget{do-all-variables-have-sensible-types}{%
\subsection{Do all variables have sensible
types?}\label{do-all-variables-have-sensible-types}}

Consider each variable and decide whether to keep, transform, or drop
it. This is a mixture of Exploratory Data Analysis and Feature
Engineering, but it's helpful to do some simple feature engineering as
you explore the data. In this project, we have all data to begin with,
so any transformations will be performed on the entire dataset. Ideally,
do the transformations as a \texttt{recipe\_step()} in the tidymodels
framework. Then the transformations would be applied to any data the
recipe was used on as part of the modeling workflow. There is less
chance of data leakage or missing a step when you perform the feature
engineering in the recipe.

\hypertarget{which-variables-to-keep-in-your-model}{%
\subsection{Which variables to keep in your
model?}\label{which-variables-to-keep-in-your-model}}

You have a number of variables and you have to decide which ones to use
in your model. For instance, you have the latitude/lognitude of the
customer, that of the merchant, the same data in radians, as well as the
\texttt{distance\_km} and \texttt{distance\_miles}. Do you need them
all?

\hypertarget{fit-your-workflows-in-smaller-sample}{%
\subsection{Fit your workflows in smaller
sample}\label{fit-your-workflows-in-smaller-sample}}

You will be running a series of different models, along the lines of the
California housing example we have seen in class. However, this dataset
has 670K rows and if you try various models and run cross validation on
them, your computer may slow down or crash.

Thus, we will work with a smaller sample of 10\% of the values the
original dataset to identify the best model, and once we have the best
model we can use the full dataset to train- test our best model.

\begin{Shaded}
\begin{Highlighting}[]
\CommentTok{\# select a smaller subset}
\NormalTok{my\_card\_fraud }\OtherTok{\textless{}{-}}\NormalTok{ card\_fraud }\SpecialCharTok{\%\textgreater{}\%} 
  \CommentTok{\# select a smaller subset, 10\% of the entire dataframe }
  \FunctionTok{slice\_sample}\NormalTok{(}\AttributeTok{prop =} \FloatTok{0.10}\NormalTok{) }\SpecialCharTok{\%\textgreater{}\%} 
  \FunctionTok{select}\NormalTok{(is\_fraud,amt,distance\_km,age,hour,category)}
\end{Highlighting}
\end{Shaded}

\hypertarget{split-the-data-in-training---testing}{%
\subsection{Split the data in training -
testing}\label{split-the-data-in-training---testing}}

\begin{Shaded}
\begin{Highlighting}[]
\CommentTok{\# **Split the data**}

\FunctionTok{set.seed}\NormalTok{(}\DecValTok{123}\NormalTok{)}

\NormalTok{data\_split }\OtherTok{\textless{}{-}} \FunctionTok{initial\_split}\NormalTok{(my\_card\_fraud, }\CommentTok{\# updated data}
                           \AttributeTok{prop =} \FloatTok{0.8}\NormalTok{, }
                           \AttributeTok{strata =}\NormalTok{ is\_fraud)}

\NormalTok{card\_fraud\_train }\OtherTok{\textless{}{-}} \FunctionTok{training}\NormalTok{(data\_split) }
\NormalTok{card\_fraud\_test }\OtherTok{\textless{}{-}} \FunctionTok{testing}\NormalTok{(data\_split)}

\CommentTok{\#Validating that amount distributes better when logaritmic}
\NormalTok{card\_fraud }\SpecialCharTok{\%\textgreater{}\%} 
  \FunctionTok{ggplot}\NormalTok{()}\SpecialCharTok{+}
  \FunctionTok{aes}\NormalTok{(}\AttributeTok{x=}\FunctionTok{log}\NormalTok{(amt))}\SpecialCharTok{+}
  \FunctionTok{geom\_histogram}\NormalTok{()}
\end{Highlighting}
\end{Shaded}

\begin{verbatim}
## `stat_bin()` using `bins = 30`. Pick better value with `binwidth`.
\end{verbatim}

\includegraphics{final_group_project_files/figure-latex/unnamed-chunk-7-1.pdf}

\hypertarget{cross-validation}{%
\subsection{Cross Validation}\label{cross-validation}}

Start with 3 CV folds to quickly get an estimate for the best model and
you can increase the number of folds to 5 or 10 later.

\begin{Shaded}
\begin{Highlighting}[]
\FunctionTok{set.seed}\NormalTok{(}\DecValTok{123}\NormalTok{)}
\NormalTok{cv\_folds }\OtherTok{\textless{}{-}} \FunctionTok{vfold\_cv}\NormalTok{(}\AttributeTok{data =}\NormalTok{ card\_fraud\_train, }
                          \AttributeTok{v =} \DecValTok{3}\NormalTok{, }
                          \AttributeTok{strata =}\NormalTok{ is\_fraud)}
\NormalTok{cv\_folds }
\end{Highlighting}
\end{Shaded}

\begin{verbatim}
## #  3-fold cross-validation using stratification 
## # A tibble: 3 x 2
##   splits                id   
##   <list>                <chr>
## 1 <split [35787/17894]> Fold1
## 2 <split [35787/17894]> Fold2
## 3 <split [35788/17893]> Fold3
\end{verbatim}

\hypertarget{define-a-tidymodels-recipe}{%
\subsection{\texorpdfstring{Define a tidymodels
\texttt{recipe}}{Define a tidymodels recipe}}\label{define-a-tidymodels-recipe}}

What steps are you going to add to your recipe? Do you need to do any
log transformations? Yes we need to do a log transformation on the
amount given the data was heavily skewed to the left. Few number of very
large transactions.

\begin{Shaded}
\begin{Highlighting}[]
\CommentTok{\# \#fraud\_rec \textless{}{-} recipe(is\_fraud \textasciitilde{} ., data = card\_fraud\_train) \%\textgreater{}\%}
\CommentTok{\#  \# step\_log(amt)\%\textgreater{}\%}
\CommentTok{\#   step\_novel(all\_nominal(), {-}all\_outcomes()) \%\textgreater{}\% \# Use before \textasciigrave{}step\_dummy()\textasciigrave{} so new level is dummified}
\CommentTok{\#   step\_dummy(hour,wday,month\_name, {-}all\_outcomes()) \%\textgreater{}\% }
\CommentTok{\#   step\_zv(is\_fraud, amt,distance\_km,age,hour,wday,month\_name, {-}all\_outcomes())  \%\textgreater{}\% }
\CommentTok{\#   step\_normalize(amt,distance\_km,age) \%\textgreater{}\% }
\CommentTok{\#   step\_naomit(everything(), skip = TRUE)}
\CommentTok{\#card\_fraud\_test2\textless{}{-}as.data.frame(card\_fraud\_test)}

\NormalTok{fraud\_rec }\OtherTok{\textless{}{-}} \FunctionTok{recipe}\NormalTok{(is\_fraud }\SpecialCharTok{\textasciitilde{}}\NormalTok{ ., }\AttributeTok{data =}\NormalTok{ card\_fraud\_train) }\SpecialCharTok{\%\textgreater{}\%}
  \FunctionTok{step\_log}\NormalTok{(amt) }\SpecialCharTok{\%\textgreater{}\%} 
  \FunctionTok{step\_naomit}\NormalTok{(}\FunctionTok{everything}\NormalTok{(), }\AttributeTok{skip =} \ConstantTok{TRUE}\NormalTok{) }\SpecialCharTok{\%\textgreater{}\%} 
  \FunctionTok{step\_novel}\NormalTok{(}\FunctionTok{all\_nominal}\NormalTok{(), }\SpecialCharTok{{-}}\FunctionTok{all\_outcomes}\NormalTok{()) }\SpecialCharTok{\%\textgreater{}\%} \CommentTok{\# Use before \textasciigrave{}step\_dummy()\textasciigrave{} so new level is dummified}
  \FunctionTok{step\_normalize}\NormalTok{(}\FunctionTok{all\_numeric}\NormalTok{()) }\SpecialCharTok{\%\textgreater{}\%} 
  \FunctionTok{step\_dummy}\NormalTok{(}\FunctionTok{all\_nominal}\NormalTok{(), }\SpecialCharTok{{-}}\FunctionTok{all\_outcomes}\NormalTok{()) }\SpecialCharTok{\%\textgreater{}\%} 
  \FunctionTok{step\_zv}\NormalTok{(}\FunctionTok{all\_numeric}\NormalTok{(), }\SpecialCharTok{{-}}\FunctionTok{all\_outcomes}\NormalTok{())}
\end{Highlighting}
\end{Shaded}

Once you have your recipe, you can check the pre-processed dataframe

\begin{Shaded}
\begin{Highlighting}[]
\NormalTok{prepped\_data }\OtherTok{\textless{}{-}} 
\NormalTok{  fraud\_rec }\SpecialCharTok{\%\textgreater{}\%} \CommentTok{\# use the recipe object}
  \FunctionTok{prep}\NormalTok{() }\SpecialCharTok{\%\textgreater{}\%} \CommentTok{\# perform the recipe on training data}
  \FunctionTok{juice}\NormalTok{() }\CommentTok{\# extract only the preprocessed dataframe }

\FunctionTok{glimpse}\NormalTok{(prepped\_data)}
\end{Highlighting}
\end{Shaded}

\begin{verbatim}
## Rows: 53,681
## Columns: 18
## $ amt                     <dbl> -1.26954983, -0.93770604, 1.69479423, -1.06754~
## $ distance_km             <dbl> 1.07940242, 2.00731983, 1.09754939, 0.76893999~
## $ age                     <dbl> 1.763960359, -0.001258131, -1.388569328, -1.72~
## $ hour                    <dbl> 1.05763722, 0.61724878, 0.32365650, 0.47045264~
## $ is_fraud                <fct> 0, 0, 0, 0, 0, 0, 0, 0, 0, 0, 0, 0, 0, 0, 0, 0~
## $ category_food_dining    <dbl> 0, 0, 0, 0, 0, 0, 0, 0, 0, 0, 0, 1, 0, 0, 0, 0~
## $ category_gas_transport  <dbl> 0, 0, 0, 0, 0, 0, 0, 0, 0, 1, 0, 0, 0, 1, 0, 0~
## $ category_grocery_net    <dbl> 0, 0, 0, 0, 0, 0, 0, 0, 0, 0, 0, 0, 0, 0, 0, 0~
## $ category_grocery_pos    <dbl> 0, 0, 0, 0, 0, 0, 1, 0, 0, 0, 0, 0, 0, 0, 1, 0~
## $ category_health_fitness <dbl> 0, 0, 0, 0, 0, 1, 0, 0, 0, 0, 0, 0, 0, 0, 0, 0~
## $ category_home           <dbl> 0, 0, 0, 0, 0, 0, 0, 0, 0, 0, 0, 0, 0, 0, 0, 1~
## $ category_kids_pets      <dbl> 0, 0, 0, 0, 0, 0, 0, 1, 0, 0, 0, 0, 0, 0, 0, 0~
## $ category_misc_net       <dbl> 0, 0, 0, 0, 0, 0, 0, 0, 0, 0, 0, 0, 0, 0, 0, 0~
## $ category_misc_pos       <dbl> 0, 0, 0, 0, 0, 0, 0, 0, 0, 0, 0, 0, 1, 0, 0, 0~
## $ category_personal_care  <dbl> 1, 0, 0, 1, 0, 0, 0, 0, 0, 0, 0, 0, 0, 0, 0, 0~
## $ category_shopping_net   <dbl> 0, 0, 1, 0, 1, 0, 0, 0, 0, 0, 0, 0, 0, 0, 0, 0~
## $ category_shopping_pos   <dbl> 0, 0, 0, 0, 0, 0, 0, 0, 0, 0, 1, 0, 0, 0, 0, 0~
## $ category_travel         <dbl> 0, 1, 0, 0, 0, 0, 0, 0, 0, 0, 0, 0, 0, 0, 0, 0~
\end{verbatim}

\hypertarget{define-various-models}{%
\subsection{Define various models}\label{define-various-models}}

You should define the following classification models:

\begin{enumerate}
\def\labelenumi{\arabic{enumi}.}
\tightlist
\item
  Logistic regression, using the \texttt{glm} engine
\item
  Decision tree, using the \texttt{C5.0} engine
\item
  Random Forest, using the \texttt{ranger} engine and setting
  \texttt{importance\ =\ "impurity"})\\
\item
  A boosted tree using Extreme Gradient Boosting, and the
  \texttt{xgboost} engine
\item
  A k-nearest neighbours, using 4 nearest\_neighbors and the
  \texttt{kknn} engine
\end{enumerate}

\begin{Shaded}
\begin{Highlighting}[]
\DocumentationTok{\#\# Model Building }

\CommentTok{\# 1. Pick a \textasciigrave{}model type\textasciigrave{}}
\CommentTok{\# 2. set the \textasciigrave{}engine\textasciigrave{}}
\CommentTok{\# 3. Set the \textasciigrave{}mode\textasciigrave{}:  classification}


\CommentTok{\# 1. Logistic regression, using the \textasciigrave{}glm\textasciigrave{} engine}

\CommentTok{\# 2. Decision tree, using the \textasciigrave{}C5.0\textasciigrave{} engine}
\CommentTok{\# 3. Random Forest, using  the \textasciigrave{}ranger\textasciigrave{} engine and setting \textasciigrave{}importance = "impurity"\textasciigrave{})  }
\CommentTok{\# 4. A boosted tree using Extreme Gradient Boosting, and the \textasciigrave{}xgboost\textasciigrave{} engine}
\CommentTok{\# 5. A k{-}nearest neighbours,  using 4 nearest\_neighbors and the \textasciigrave{}kknn\textasciigrave{} engine }
\CommentTok{\# Logistic regression}
\NormalTok{log\_spec }\OtherTok{\textless{}{-}}  \FunctionTok{logistic\_reg}\NormalTok{() }\SpecialCharTok{\%\textgreater{}\%}  \CommentTok{\# model type}
  \FunctionTok{set\_engine}\NormalTok{(}\AttributeTok{engine =} \StringTok{"glm"}\NormalTok{) }\SpecialCharTok{\%\textgreater{}\%}  \CommentTok{\# model engine}
  \FunctionTok{set\_mode}\NormalTok{(}\StringTok{"classification"}\NormalTok{) }\CommentTok{\# model mode}

\CommentTok{\# Show your model specification}
\NormalTok{log\_spec}
\end{Highlighting}
\end{Shaded}

\begin{verbatim}
## Logistic Regression Model Specification (classification)
## 
## Computational engine: glm
\end{verbatim}

\begin{Shaded}
\begin{Highlighting}[]
\CommentTok{\# Decision Tree}
\NormalTok{tree\_spec }\OtherTok{\textless{}{-}} \FunctionTok{decision\_tree}\NormalTok{() }\SpecialCharTok{\%\textgreater{}\%}
  \FunctionTok{set\_engine}\NormalTok{(}\AttributeTok{engine =} \StringTok{"C5.0"}\NormalTok{) }\SpecialCharTok{\%\textgreater{}\%}
  \FunctionTok{set\_mode}\NormalTok{(}\StringTok{"classification"}\NormalTok{)}

\NormalTok{tree\_spec}
\end{Highlighting}
\end{Shaded}

\begin{verbatim}
## Decision Tree Model Specification (classification)
## 
## Computational engine: C5.0
\end{verbatim}

\begin{Shaded}
\begin{Highlighting}[]
\CommentTok{\# Random Forest}
\FunctionTok{library}\NormalTok{(ranger)}

\NormalTok{rf\_spec }\OtherTok{\textless{}{-}} 
  \FunctionTok{rand\_forest}\NormalTok{() }\SpecialCharTok{\%\textgreater{}\%} 
  \FunctionTok{set\_engine}\NormalTok{(}\StringTok{"ranger"}\NormalTok{, }\AttributeTok{importance =} \StringTok{"impurity"}\NormalTok{) }\SpecialCharTok{\%\textgreater{}\%} 
  \FunctionTok{set\_mode}\NormalTok{(}\StringTok{"classification"}\NormalTok{)}


\CommentTok{\# Boosted tree (XGBoost)}
\FunctionTok{library}\NormalTok{(xgboost)}
\end{Highlighting}
\end{Shaded}

\begin{verbatim}
## 
## Attaching package: 'xgboost'
\end{verbatim}

\begin{verbatim}
## The following object is masked from 'package:dplyr':
## 
##     slice
\end{verbatim}

\begin{Shaded}
\begin{Highlighting}[]
\NormalTok{xgb\_spec }\OtherTok{\textless{}{-}} 
  \FunctionTok{boost\_tree}\NormalTok{() }\SpecialCharTok{\%\textgreater{}\%} 
  \FunctionTok{set\_engine}\NormalTok{(}\StringTok{"xgboost"}\NormalTok{) }\SpecialCharTok{\%\textgreater{}\%} 
  \FunctionTok{set\_mode}\NormalTok{(}\StringTok{"classification"}\NormalTok{) }

\CommentTok{\# K{-}nearest neighbour (k{-}NN)}
\NormalTok{knn\_spec }\OtherTok{\textless{}{-}} 
  \FunctionTok{nearest\_neighbor}\NormalTok{(}\AttributeTok{neighbors =} \DecValTok{4}\NormalTok{) }\SpecialCharTok{\%\textgreater{}\%} \CommentTok{\# we can adjust the number of neighbors }
  \FunctionTok{set\_engine}\NormalTok{(}\StringTok{"kknn"}\NormalTok{) }\SpecialCharTok{\%\textgreater{}\%} 
  \FunctionTok{set\_mode}\NormalTok{(}\StringTok{"classification"}\NormalTok{)}
\end{Highlighting}
\end{Shaded}

\hypertarget{bundle-recipe-and-model-with-workflows}{%
\subsection{\texorpdfstring{Bundle recipe and model with
\texttt{workflows}}{Bundle recipe and model with workflows}}\label{bundle-recipe-and-model-with-workflows}}

\begin{Shaded}
\begin{Highlighting}[]
\NormalTok{log\_wflow }\OtherTok{\textless{}{-}} \CommentTok{\# new workflow object}
 \FunctionTok{workflow}\NormalTok{() }\SpecialCharTok{\%\textgreater{}\%} \CommentTok{\# use workflow function}
 \FunctionTok{add\_recipe}\NormalTok{(fraud\_rec) }\SpecialCharTok{\%\textgreater{}\%}   \CommentTok{\# use the new recipe}
 \FunctionTok{add\_model}\NormalTok{(log\_spec)   }\CommentTok{\# add your model spec}

\CommentTok{\# show object}
\NormalTok{log\_wflow}
\end{Highlighting}
\end{Shaded}

\begin{verbatim}
## == Workflow ====================================================================
## Preprocessor: Recipe
## Model: logistic_reg()
## 
## -- Preprocessor ----------------------------------------------------------------
## 6 Recipe Steps
## 
## * step_log()
## * step_naomit()
## * step_novel()
## * step_normalize()
## * step_dummy()
## * step_zv()
## 
## -- Model -----------------------------------------------------------------------
## Logistic Regression Model Specification (classification)
## 
## Computational engine: glm
\end{verbatim}

\begin{Shaded}
\begin{Highlighting}[]
\DocumentationTok{\#\# A few more workflows}

\NormalTok{tree\_wflow }\OtherTok{\textless{}{-}}
 \FunctionTok{workflow}\NormalTok{() }\SpecialCharTok{\%\textgreater{}\%}
 \FunctionTok{add\_recipe}\NormalTok{(fraud\_rec) }\SpecialCharTok{\%\textgreater{}\%} 
 \FunctionTok{add\_model}\NormalTok{(tree\_spec) }

\NormalTok{rf\_wflow }\OtherTok{\textless{}{-}}
 \FunctionTok{workflow}\NormalTok{() }\SpecialCharTok{\%\textgreater{}\%}
 \FunctionTok{add\_recipe}\NormalTok{(fraud\_rec) }\SpecialCharTok{\%\textgreater{}\%} 
 \FunctionTok{add\_model}\NormalTok{(rf\_spec) }

\NormalTok{xgb\_wflow }\OtherTok{\textless{}{-}}
 \FunctionTok{workflow}\NormalTok{() }\SpecialCharTok{\%\textgreater{}\%}
 \FunctionTok{add\_recipe}\NormalTok{(fraud\_rec) }\SpecialCharTok{\%\textgreater{}\%} 
 \FunctionTok{add\_model}\NormalTok{(xgb\_spec)}

\NormalTok{knn\_wflow }\OtherTok{\textless{}{-}}
 \FunctionTok{workflow}\NormalTok{() }\SpecialCharTok{\%\textgreater{}\%}
 \FunctionTok{add\_recipe}\NormalTok{(fraud\_rec) }\SpecialCharTok{\%\textgreater{}\%} 
 \FunctionTok{add\_model}\NormalTok{(knn\_spec)}

\DocumentationTok{\#\# Bundle recipe and model with \textasciigrave{}workflows\textasciigrave{}}


\NormalTok{log\_wflow }\OtherTok{\textless{}{-}} \CommentTok{\# new workflow object}
 \FunctionTok{workflow}\NormalTok{() }\SpecialCharTok{\%\textgreater{}\%} \CommentTok{\# use workflow function}
 \FunctionTok{add\_recipe}\NormalTok{(fraud\_rec) }\SpecialCharTok{\%\textgreater{}\%}   \CommentTok{\# use the new recipe}
 \FunctionTok{add\_model}\NormalTok{(log\_spec)   }\CommentTok{\# add your model spec}
\end{Highlighting}
\end{Shaded}

\hypertarget{fit-models}{%
\subsection{Fit models}\label{fit-models}}

You may want to compare the time it takes to fit each model.
\texttt{tic()} starts a simple timer and \texttt{toc()} stops it

\begin{Shaded}
\begin{Highlighting}[]
\FunctionTok{tic}\NormalTok{()}
\NormalTok{log\_res }\OtherTok{\textless{}{-}}\NormalTok{ log\_wflow }\SpecialCharTok{\%\textgreater{}\%} 
  \FunctionTok{fit\_resamples}\NormalTok{(}
    \AttributeTok{resamples =}\NormalTok{ cv\_folds, }
    \AttributeTok{metrics =} \FunctionTok{metric\_set}\NormalTok{(}
\NormalTok{      recall, precision, f\_meas, accuracy,}
\NormalTok{      kap, roc\_auc, sens, spec),}
    \AttributeTok{control =} \FunctionTok{control\_resamples}\NormalTok{(}\AttributeTok{save\_pred =} \ConstantTok{TRUE}\NormalTok{)) }
\NormalTok{time }\OtherTok{\textless{}{-}} \FunctionTok{toc}\NormalTok{()}
\end{Highlighting}
\end{Shaded}

\begin{verbatim}
## 1.28 sec elapsed
\end{verbatim}

\begin{Shaded}
\begin{Highlighting}[]
\NormalTok{log\_time }\OtherTok{\textless{}{-}}\NormalTok{ time[[}\DecValTok{4}\NormalTok{]]}

\DocumentationTok{\#\# Evaluate Models}

\DocumentationTok{\#\# Logistic regression results\{.smaller\}}

\NormalTok{log\_res }\OtherTok{\textless{}{-}}\NormalTok{ log\_wflow }\SpecialCharTok{\%\textgreater{}\%} 
  \FunctionTok{fit\_resamples}\NormalTok{(}
    \AttributeTok{resamples =}\NormalTok{ cv\_folds, }
    \AttributeTok{metrics =} \FunctionTok{metric\_set}\NormalTok{(}
\NormalTok{      recall, precision, f\_meas, accuracy,}
\NormalTok{      kap, roc\_auc, sens, spec),}
    \AttributeTok{control =} \FunctionTok{control\_resamples}\NormalTok{(}\AttributeTok{save\_pred =} \ConstantTok{TRUE}\NormalTok{)) }

\CommentTok{\# Show average performance over all folds (note that we use log\_res):}
\NormalTok{log\_res }\SpecialCharTok{\%\textgreater{}\%}  \FunctionTok{collect\_metrics}\NormalTok{(}\AttributeTok{summarize =} \ConstantTok{TRUE}\NormalTok{)}
\end{Highlighting}
\end{Shaded}

\begin{verbatim}
## # A tibble: 8 x 6
##   .metric   .estimator  mean     n   std_err .config             
##   <chr>     <chr>      <dbl> <int>     <dbl> <chr>               
## 1 accuracy  binary     0.994     3 0.000287  Preprocessor1_Model1
## 2 f_meas    binary     0.178     3 0.0562    Preprocessor1_Model1
## 3 kap       binary     0.177     3 0.0559    Preprocessor1_Model1
## 4 precision binary     0.668     3 0.0558    Preprocessor1_Model1
## 5 recall    binary     0.106     3 0.0366    Preprocessor1_Model1
## 6 roc_auc   binary     0.851     3 0.0111    Preprocessor1_Model1
## 7 sens      binary     0.106     3 0.0366    Preprocessor1_Model1
## 8 spec      binary     1.00      3 0.0000858 Preprocessor1_Model1
\end{verbatim}

\begin{Shaded}
\begin{Highlighting}[]
\CommentTok{\# Show performance for every single fold:}
\NormalTok{log\_res }\SpecialCharTok{\%\textgreater{}\%}  \FunctionTok{collect\_metrics}\NormalTok{(}\AttributeTok{summarize =} \ConstantTok{FALSE}\NormalTok{)}
\end{Highlighting}
\end{Shaded}

\begin{verbatim}
## # A tibble: 24 x 5
##    id    .metric   .estimator .estimate .config             
##    <chr> <chr>     <chr>          <dbl> <chr>               
##  1 Fold1 recall    binary        0.0388 Preprocessor1_Model1
##  2 Fold1 precision binary        0.571  Preprocessor1_Model1
##  3 Fold1 f_meas    binary        0.0727 Preprocessor1_Model1
##  4 Fold1 accuracy  binary        0.994  Preprocessor1_Model1
##  5 Fold1 kap       binary        0.0720 Preprocessor1_Model1
##  6 Fold1 sens      binary        0.0388 Preprocessor1_Model1
##  7 Fold1 spec      binary        1.00   Preprocessor1_Model1
##  8 Fold1 roc_auc   binary        0.871  Preprocessor1_Model1
##  9 Fold2 recall    binary        0.165  Preprocessor1_Model1
## 10 Fold2 precision binary        0.667  Preprocessor1_Model1
## # i 14 more rows
\end{verbatim}

\begin{Shaded}
\begin{Highlighting}[]
\DocumentationTok{\#\# \textasciigrave{}collect\_predictions()\textasciigrave{} and get confusion matrix\{.smaller\}}

\NormalTok{log\_pred }\OtherTok{\textless{}{-}}\NormalTok{ log\_res }\SpecialCharTok{\%\textgreater{}\%} \FunctionTok{collect\_predictions}\NormalTok{()}

\NormalTok{log\_pred }\SpecialCharTok{\%\textgreater{}\%}  \FunctionTok{conf\_mat}\NormalTok{(is\_fraud, .pred\_class) }
\end{Highlighting}
\end{Shaded}

\begin{verbatim}
##           Truth
## Prediction     1     0
##          1    33    15
##          0   282 53351
\end{verbatim}

\begin{Shaded}
\begin{Highlighting}[]
\NormalTok{log\_pred }\SpecialCharTok{\%\textgreater{}\%} 
  \FunctionTok{conf\_mat}\NormalTok{(is\_fraud, .pred\_class) }\SpecialCharTok{\%\textgreater{}\%} 
  \FunctionTok{autoplot}\NormalTok{(}\AttributeTok{type =} \StringTok{"mosaic"}\NormalTok{) }\SpecialCharTok{+}
  \FunctionTok{geom\_label}\NormalTok{(}\FunctionTok{aes}\NormalTok{(}
      \AttributeTok{x =}\NormalTok{ (xmax }\SpecialCharTok{+}\NormalTok{ xmin) }\SpecialCharTok{/} \DecValTok{2}\NormalTok{, }
      \AttributeTok{y =}\NormalTok{ (ymax }\SpecialCharTok{+}\NormalTok{ ymin) }\SpecialCharTok{/} \DecValTok{2}\NormalTok{, }
      \AttributeTok{label =} \FunctionTok{c}\NormalTok{(}\StringTok{"TP"}\NormalTok{, }\StringTok{"FN"}\NormalTok{, }\StringTok{"FP"}\NormalTok{, }\StringTok{"TN"}\NormalTok{)))}
\end{Highlighting}
\end{Shaded}

\includegraphics{final_group_project_files/figure-latex/unnamed-chunk-10-1.pdf}

\begin{Shaded}
\begin{Highlighting}[]
\NormalTok{log\_pred }\SpecialCharTok{\%\textgreater{}\%} 
  \FunctionTok{conf\_mat}\NormalTok{(is\_fraud, .pred\_class) }\SpecialCharTok{\%\textgreater{}\%} 
  \FunctionTok{autoplot}\NormalTok{(}\AttributeTok{type =} \StringTok{"heatmap"}\NormalTok{)}
\end{Highlighting}
\end{Shaded}

\includegraphics{final_group_project_files/figure-latex/unnamed-chunk-10-2.pdf}

\begin{Shaded}
\begin{Highlighting}[]
\DocumentationTok{\#\# ROC Curve}

\NormalTok{log\_pred }\SpecialCharTok{\%\textgreater{}\%} 
  \FunctionTok{group\_by}\NormalTok{(id) }\SpecialCharTok{\%\textgreater{}\%} \CommentTok{\# id contains our folds}
  \FunctionTok{roc\_curve}\NormalTok{(is\_fraud, .pred\_1) }\SpecialCharTok{\%\textgreater{}\%} 
  \FunctionTok{autoplot}\NormalTok{()}
\end{Highlighting}
\end{Shaded}

\includegraphics{final_group_project_files/figure-latex/unnamed-chunk-10-3.pdf}

\begin{Shaded}
\begin{Highlighting}[]
\DocumentationTok{\#\# Decision Tree results}

\NormalTok{tree\_res }\OtherTok{\textless{}{-}}
\NormalTok{  tree\_wflow }\SpecialCharTok{\%\textgreater{}\%} 
  \FunctionTok{fit\_resamples}\NormalTok{(}
    \AttributeTok{resamples =}\NormalTok{ cv\_folds, }
    \AttributeTok{metrics =} \FunctionTok{metric\_set}\NormalTok{(}
\NormalTok{      recall, precision, f\_meas, }
\NormalTok{      accuracy, kap,}
\NormalTok{      roc\_auc, sens, spec),}
    \AttributeTok{control =} \FunctionTok{control\_resamples}\NormalTok{(}\AttributeTok{save\_pred =} \ConstantTok{TRUE}\NormalTok{)}
\NormalTok{    ) }

\NormalTok{tree\_res }\SpecialCharTok{\%\textgreater{}\%}  \FunctionTok{collect\_metrics}\NormalTok{(}\AttributeTok{summarize =} \ConstantTok{TRUE}\NormalTok{)}
\end{Highlighting}
\end{Shaded}

\begin{verbatim}
## # A tibble: 8 x 6
##   .metric   .estimator  mean     n   std_err .config             
##   <chr>     <chr>      <dbl> <int>     <dbl> <chr>               
## 1 accuracy  binary     0.997     3 0.000541  Preprocessor1_Model1
## 2 f_meas    binary     0.744     3 0.0474    Preprocessor1_Model1
## 3 kap       binary     0.743     3 0.0476    Preprocessor1_Model1
## 4 precision binary     0.857     3 0.0219    Preprocessor1_Model1
## 5 recall    binary     0.660     3 0.0608    Preprocessor1_Model1
## 6 roc_auc   binary     0.906     3 0.0222    Preprocessor1_Model1
## 7 sens      binary     0.660     3 0.0608    Preprocessor1_Model1
## 8 spec      binary     0.999     3 0.0000819 Preprocessor1_Model1
\end{verbatim}

\begin{Shaded}
\begin{Highlighting}[]
\DocumentationTok{\#\# Random Forest}

\NormalTok{rf\_res }\OtherTok{\textless{}{-}}
\NormalTok{  rf\_wflow }\SpecialCharTok{\%\textgreater{}\%} 
  \FunctionTok{fit\_resamples}\NormalTok{(}
    \AttributeTok{resamples =}\NormalTok{ cv\_folds, }
    \AttributeTok{metrics =} \FunctionTok{metric\_set}\NormalTok{(}
\NormalTok{      recall, precision, f\_meas, }
\NormalTok{      accuracy, kap,}
\NormalTok{      roc\_auc, sens, spec),}
    \AttributeTok{control =} \FunctionTok{control\_resamples}\NormalTok{(}\AttributeTok{save\_pred =} \ConstantTok{TRUE}\NormalTok{)}
\NormalTok{    ) }

\NormalTok{rf\_res }\SpecialCharTok{\%\textgreater{}\%}  \FunctionTok{collect\_metrics}\NormalTok{(}\AttributeTok{summarize =} \ConstantTok{TRUE}\NormalTok{)}
\end{Highlighting}
\end{Shaded}

\begin{verbatim}
## # A tibble: 8 x 6
##   .metric   .estimator  mean     n   std_err .config             
##   <chr>     <chr>      <dbl> <int>     <dbl> <chr>               
## 1 accuracy  binary     0.997     3 0.000372  Preprocessor1_Model1
## 2 f_meas    binary     0.679     3 0.0331    Preprocessor1_Model1
## 3 kap       binary     0.678     3 0.0332    Preprocessor1_Model1
## 4 precision binary     0.971     3 0.0150    Preprocessor1_Model1
## 5 recall    binary     0.525     3 0.0394    Preprocessor1_Model1
## 6 roc_auc   binary     0.965     3 0.00339   Preprocessor1_Model1
## 7 sens      binary     0.525     3 0.0394    Preprocessor1_Model1
## 8 spec      binary     1.00      3 0.0000496 Preprocessor1_Model1
\end{verbatim}

\begin{Shaded}
\begin{Highlighting}[]
\DocumentationTok{\#\# Boosted tree {-} XGBoost}

\NormalTok{xgb\_res }\OtherTok{\textless{}{-}} 
\NormalTok{  xgb\_wflow }\SpecialCharTok{\%\textgreater{}\%} 
  \FunctionTok{fit\_resamples}\NormalTok{(}
    \AttributeTok{resamples =}\NormalTok{ cv\_folds, }
    \AttributeTok{metrics =} \FunctionTok{metric\_set}\NormalTok{(}
\NormalTok{      recall, precision, f\_meas, }
\NormalTok{      accuracy, kap,}
\NormalTok{      roc\_auc, sens, spec),}
    \AttributeTok{control =} \FunctionTok{control\_resamples}\NormalTok{(}\AttributeTok{save\_pred =} \ConstantTok{TRUE}\NormalTok{)}
\NormalTok{    ) }

\NormalTok{xgb\_res }\SpecialCharTok{\%\textgreater{}\%} \FunctionTok{collect\_metrics}\NormalTok{(}\AttributeTok{summarize =} \ConstantTok{TRUE}\NormalTok{)}
\end{Highlighting}
\end{Shaded}

\begin{verbatim}
## # A tibble: 8 x 6
##   .metric   .estimator  mean     n   std_err .config             
##   <chr>     <chr>      <dbl> <int>     <dbl> <chr>               
## 1 accuracy  binary     0.998     3 0.000271  Preprocessor1_Model1
## 2 f_meas    binary     0.795     3 0.0201    Preprocessor1_Model1
## 3 kap       binary     0.794     3 0.0202    Preprocessor1_Model1
## 4 precision binary     0.910     3 0.00491   Preprocessor1_Model1
## 5 recall    binary     0.708     3 0.0322    Preprocessor1_Model1
## 6 roc_auc   binary     0.971     3 0.00728   Preprocessor1_Model1
## 7 sens      binary     0.708     3 0.0322    Preprocessor1_Model1
## 8 spec      binary     1.00      3 0.0000186 Preprocessor1_Model1
\end{verbatim}

\begin{Shaded}
\begin{Highlighting}[]
\DocumentationTok{\#\# K{-}nearest neighbour}

\NormalTok{knn\_res }\OtherTok{\textless{}{-}} 
\NormalTok{  knn\_wflow }\SpecialCharTok{\%\textgreater{}\%} 
  \FunctionTok{fit\_resamples}\NormalTok{(}
    \AttributeTok{resamples =}\NormalTok{ cv\_folds, }
    \AttributeTok{metrics =} \FunctionTok{metric\_set}\NormalTok{(}
\NormalTok{      recall, precision, f\_meas, }
\NormalTok{      accuracy, kap,}
\NormalTok{      roc\_auc, sens, spec),}
    \AttributeTok{control =} \FunctionTok{control\_resamples}\NormalTok{(}\AttributeTok{save\_pred =} \ConstantTok{TRUE}\NormalTok{)}
\NormalTok{    ) }

\NormalTok{knn\_res }\SpecialCharTok{\%\textgreater{}\%} \FunctionTok{collect\_metrics}\NormalTok{(}\AttributeTok{summarize =} \ConstantTok{TRUE}\NormalTok{)}
\end{Highlighting}
\end{Shaded}

\begin{verbatim}
## # A tibble: 8 x 6
##   .metric   .estimator  mean     n  std_err .config             
##   <chr>     <chr>      <dbl> <int>    <dbl> <chr>               
## 1 accuracy  binary     0.996     3 0.000227 Preprocessor1_Model1
## 2 f_meas    binary     0.603     3 0.00682  Preprocessor1_Model1
## 3 kap       binary     0.600     3 0.00681  Preprocessor1_Model1
## 4 precision binary     0.683     3 0.00488  Preprocessor1_Model1
## 5 recall    binary     0.539     3 0.0126   Preprocessor1_Model1
## 6 roc_auc   binary     0.838     3 0.0103   Preprocessor1_Model1
## 7 sens      binary     0.539     3 0.0126   Preprocessor1_Model1
## 8 spec      binary     0.999     3 0.000132 Preprocessor1_Model1
\end{verbatim}

\begin{Shaded}
\begin{Highlighting}[]
\NormalTok{log\_metrics }\OtherTok{\textless{}{-}} 
\NormalTok{  log\_res }\SpecialCharTok{\%\textgreater{}\%} 
  \FunctionTok{collect\_metrics}\NormalTok{(}\AttributeTok{summarise =} \ConstantTok{TRUE}\NormalTok{) }\SpecialCharTok{\%\textgreater{}\%}
  \CommentTok{\# add the name of the model to every row}
  \FunctionTok{mutate}\NormalTok{(}\AttributeTok{model =} \StringTok{"Logistic Regression"}\NormalTok{) }

\NormalTok{tree\_metrics }\OtherTok{\textless{}{-}} 
\NormalTok{  tree\_res }\SpecialCharTok{\%\textgreater{}\%} 
  \FunctionTok{collect\_metrics}\NormalTok{(}\AttributeTok{summarise =} \ConstantTok{TRUE}\NormalTok{) }\SpecialCharTok{\%\textgreater{}\%}
  \FunctionTok{mutate}\NormalTok{(}\AttributeTok{model =} \StringTok{"Decision Tree"}\NormalTok{)}

\NormalTok{rf\_metrics }\OtherTok{\textless{}{-}} 
\NormalTok{  rf\_res }\SpecialCharTok{\%\textgreater{}\%} 
  \FunctionTok{collect\_metrics}\NormalTok{(}\AttributeTok{summarise =} \ConstantTok{TRUE}\NormalTok{) }\SpecialCharTok{\%\textgreater{}\%}
  \FunctionTok{mutate}\NormalTok{(}\AttributeTok{model =} \StringTok{"Random Forest"}\NormalTok{)}

\NormalTok{xgb\_metrics }\OtherTok{\textless{}{-}} 
\NormalTok{  xgb\_res }\SpecialCharTok{\%\textgreater{}\%} 
  \FunctionTok{collect\_metrics}\NormalTok{(}\AttributeTok{summarise =} \ConstantTok{TRUE}\NormalTok{) }\SpecialCharTok{\%\textgreater{}\%}
  \FunctionTok{mutate}\NormalTok{(}\AttributeTok{model =} \StringTok{"XGBoost"}\NormalTok{)}

\NormalTok{knn\_metrics }\OtherTok{\textless{}{-}} 
\NormalTok{  knn\_res }\SpecialCharTok{\%\textgreater{}\%} 
  \FunctionTok{collect\_metrics}\NormalTok{(}\AttributeTok{summarise =} \ConstantTok{TRUE}\NormalTok{) }\SpecialCharTok{\%\textgreater{}\%}
  \FunctionTok{mutate}\NormalTok{(}\AttributeTok{model =} \StringTok{"Knn"}\NormalTok{)}

\CommentTok{\# create dataframe with all models}
\NormalTok{model\_compare }\OtherTok{\textless{}{-}} \FunctionTok{bind\_rows}\NormalTok{(log\_metrics,}
\NormalTok{                           tree\_metrics,}
\NormalTok{                           rf\_metrics,}
\NormalTok{                           xgb\_metrics,}
\NormalTok{                           knn\_metrics)}
\CommentTok{\#Pivot wider to create barplot}
\NormalTok{  model\_comp }\OtherTok{\textless{}{-}}\NormalTok{ model\_compare }\SpecialCharTok{\%\textgreater{}\%} 
  \FunctionTok{select}\NormalTok{(model, .metric, mean, std\_err) }\SpecialCharTok{\%\textgreater{}\%} 
  \FunctionTok{pivot\_wider}\NormalTok{(}\AttributeTok{names\_from =}\NormalTok{ .metric, }\AttributeTok{values\_from =} \FunctionTok{c}\NormalTok{(mean, std\_err)) }

\CommentTok{\# show mean are under the curve (ROC{-}AUC) for every model}
\NormalTok{model\_comp }\SpecialCharTok{\%\textgreater{}\%} 
  \FunctionTok{arrange}\NormalTok{(mean\_roc\_auc) }\SpecialCharTok{\%\textgreater{}\%} 
  \FunctionTok{mutate}\NormalTok{(}\AttributeTok{model =} \FunctionTok{fct\_reorder}\NormalTok{(model, mean\_roc\_auc)) }\SpecialCharTok{\%\textgreater{}\%} \CommentTok{\# order results}
  \FunctionTok{ggplot}\NormalTok{(}\FunctionTok{aes}\NormalTok{(model, mean\_roc\_auc, }\AttributeTok{fill=}\NormalTok{model)) }\SpecialCharTok{+}
  \FunctionTok{geom\_col}\NormalTok{() }\SpecialCharTok{+}
  \FunctionTok{coord\_flip}\NormalTok{() }\SpecialCharTok{+}
  \FunctionTok{scale\_fill\_brewer}\NormalTok{(}\AttributeTok{palette =} \StringTok{"Blues"}\NormalTok{) }\SpecialCharTok{+}
   \FunctionTok{geom\_text}\NormalTok{(}
     \AttributeTok{size =} \DecValTok{3}\NormalTok{,}
     \FunctionTok{aes}\NormalTok{(}\AttributeTok{label =} \FunctionTok{round}\NormalTok{(mean\_roc\_auc, }\DecValTok{2}\NormalTok{), }
         \AttributeTok{y =}\NormalTok{ mean\_roc\_auc }\SpecialCharTok{+} \FloatTok{0.08}\NormalTok{),}
     \AttributeTok{vjust =} \DecValTok{1}
\NormalTok{  )}\SpecialCharTok{+}
  \FunctionTok{theme\_light}\NormalTok{()}\SpecialCharTok{+}
  \FunctionTok{theme}\NormalTok{(}\AttributeTok{legend.position =} \StringTok{"none"}\NormalTok{)}\SpecialCharTok{+}
  \FunctionTok{labs}\NormalTok{(}\AttributeTok{y =} \ConstantTok{NULL}\NormalTok{)}
\end{Highlighting}
\end{Shaded}

\includegraphics{final_group_project_files/figure-latex/unnamed-chunk-15-1.pdf}
\#\# Which metric to use

This is a highly imbalanced data set, as roughly 99.5\% of all
transactions are ok, and it's only 0.5\% of transactions that are
fraudulent. A \texttt{naive} model, which classifies everything as ok
and not-fraud, would have an accuracy of 99.5\%, but what about the
sensitivity, specificity, the AUC, etc?

\hypertarget{last_fit}{%
\subsection{`last\_fit()}\label{last_fit}}

\begin{Shaded}
\begin{Highlighting}[]
\DocumentationTok{\#\# \textasciigrave{}last\_fit()\textasciigrave{} on test set}

\CommentTok{\# {-} \textasciigrave{}last\_fit()\textasciigrave{}  fits a model to the whole training data and evaluates it on the test set. }
\CommentTok{\# {-} provide the workflow object of the best model as well as the data split object (not the training data). }
 
\NormalTok{last\_fit\_xgb }\OtherTok{\textless{}{-}} \FunctionTok{last\_fit}\NormalTok{(xgb\_wflow, }
                        \AttributeTok{split =}\NormalTok{ data\_split,}
                        \AttributeTok{metrics =} \FunctionTok{metric\_set}\NormalTok{(}
\NormalTok{                          accuracy, f\_meas, kap, precision,}
\NormalTok{                          recall, roc\_auc, sens, spec))}

\NormalTok{last\_fit\_xgb }\SpecialCharTok{\%\textgreater{}\%} \FunctionTok{collect\_metrics}\NormalTok{(}\AttributeTok{summarize =} \ConstantTok{TRUE}\NormalTok{)}
\end{Highlighting}
\end{Shaded}

\begin{verbatim}
## # A tibble: 8 x 4
##   .metric   .estimator .estimate .config             
##   <chr>     <chr>          <dbl> <chr>               
## 1 accuracy  binary         0.998 Preprocessor1_Model1
## 2 f_meas    binary         0.781 Preprocessor1_Model1
## 3 kap       binary         0.780 Preprocessor1_Model1
## 4 precision binary         0.919 Preprocessor1_Model1
## 5 recall    binary         0.679 Preprocessor1_Model1
## 6 sens      binary         0.679 Preprocessor1_Model1
## 7 spec      binary         1.00  Preprocessor1_Model1
## 8 roc_auc   binary         0.985 Preprocessor1_Model1
\end{verbatim}

\begin{Shaded}
\begin{Highlighting}[]
\CommentTok{\#Compare to training}
\NormalTok{xgb\_res }\SpecialCharTok{\%\textgreater{}\%} \FunctionTok{collect\_metrics}\NormalTok{(}\AttributeTok{summarize =} \ConstantTok{TRUE}\NormalTok{)}
\end{Highlighting}
\end{Shaded}

\begin{verbatim}
## # A tibble: 8 x 6
##   .metric   .estimator  mean     n   std_err .config             
##   <chr>     <chr>      <dbl> <int>     <dbl> <chr>               
## 1 accuracy  binary     0.998     3 0.000271  Preprocessor1_Model1
## 2 f_meas    binary     0.795     3 0.0201    Preprocessor1_Model1
## 3 kap       binary     0.794     3 0.0202    Preprocessor1_Model1
## 4 precision binary     0.910     3 0.00491   Preprocessor1_Model1
## 5 recall    binary     0.708     3 0.0322    Preprocessor1_Model1
## 6 roc_auc   binary     0.971     3 0.00728   Preprocessor1_Model1
## 7 sens      binary     0.708     3 0.0322    Preprocessor1_Model1
## 8 spec      binary     1.00      3 0.0000186 Preprocessor1_Model1
\end{verbatim}

\hypertarget{get-variable-importance-using-vip-package}{%
\subsection{\texorpdfstring{Get variable importance using \texttt{vip}
package}{Get variable importance using vip package}}\label{get-variable-importance-using-vip-package}}

\begin{Shaded}
\begin{Highlighting}[]
\DocumentationTok{\#\# Variable importance using \textasciigrave{}\{vip\}\textasciigrave{} package}

\FunctionTok{library}\NormalTok{(vip)}

\NormalTok{last\_fit\_xgb }\SpecialCharTok{\%\textgreater{}\%} 
  \FunctionTok{pluck}\NormalTok{(}\StringTok{".workflow"}\NormalTok{, }\DecValTok{1}\NormalTok{) }\SpecialCharTok{\%\textgreater{}\%}   
  \FunctionTok{pull\_workflow\_fit}\NormalTok{() }\SpecialCharTok{\%\textgreater{}\%} 
  \FunctionTok{vip}\NormalTok{(}\AttributeTok{num\_features =} \DecValTok{10}\NormalTok{) }\SpecialCharTok{+}
  \FunctionTok{theme\_light}\NormalTok{()}
\end{Highlighting}
\end{Shaded}

\begin{verbatim}
## Warning: `pull_workflow_fit()` was deprecated in workflows 0.2.3.
## i Please use `extract_fit_parsnip()` instead.
## This warning is displayed once every 8 hours.
## Call `lifecycle::last_lifecycle_warnings()` to see where this warning was
## generated.
\end{verbatim}

\includegraphics{final_group_project_files/figure-latex/unnamed-chunk-18-1.pdf}
\#\# Plot Final Confusion matrix and ROC curve

\begin{Shaded}
\begin{Highlighting}[]
\DocumentationTok{\#\# Final Confusion Matrix}

\NormalTok{last\_fit\_xgb }\SpecialCharTok{\%\textgreater{}\%}
  \FunctionTok{collect\_predictions}\NormalTok{() }\SpecialCharTok{\%\textgreater{}\%} 
  \FunctionTok{conf\_mat}\NormalTok{(is\_fraud, .pred\_class) }\SpecialCharTok{\%\textgreater{}\%} 
  \FunctionTok{autoplot}\NormalTok{(}\AttributeTok{type =} \StringTok{"heatmap"}\NormalTok{)}
\end{Highlighting}
\end{Shaded}

\includegraphics{final_group_project_files/figure-latex/unnamed-chunk-19-1.pdf}

\begin{Shaded}
\begin{Highlighting}[]
\DocumentationTok{\#\# Final ROC curve}
\NormalTok{last\_fit\_xgb }\SpecialCharTok{\%\textgreater{}\%} 
  \FunctionTok{collect\_predictions}\NormalTok{() }\SpecialCharTok{\%\textgreater{}\%} 
  \FunctionTok{roc\_curve}\NormalTok{(is\_fraud, .pred\_1) }\SpecialCharTok{\%\textgreater{}\%}
  \FunctionTok{autoplot}\NormalTok{()}
\end{Highlighting}
\end{Shaded}

\includegraphics{final_group_project_files/figure-latex/unnamed-chunk-20-1.pdf}

\hypertarget{compare-models}{%
\subsection{Compare models}\label{compare-models}}

\begin{Shaded}
\begin{Highlighting}[]
\DocumentationTok{\#\# Model Comparison}

\NormalTok{log\_metrics }\OtherTok{\textless{}{-}} 
\NormalTok{  log\_res }\SpecialCharTok{\%\textgreater{}\%} 
  \FunctionTok{collect\_metrics}\NormalTok{(}\AttributeTok{summarise =} \ConstantTok{TRUE}\NormalTok{) }\SpecialCharTok{\%\textgreater{}\%}
  \CommentTok{\# add the name of the model to every row}
  \FunctionTok{mutate}\NormalTok{(}\AttributeTok{model =} \StringTok{"Logistic Regression"}\NormalTok{,}
         \AttributeTok{time =}\NormalTok{ log\_time)}

\CommentTok{\# add mode models here}

\CommentTok{\# create dataframe with all models}
\NormalTok{model\_compare }\OtherTok{\textless{}{-}} \FunctionTok{bind\_rows}\NormalTok{(log\_metrics,}
\NormalTok{                            tree\_metrics,}
\NormalTok{                            rf\_metrics,}
\NormalTok{                           xgb\_metrics,}
\NormalTok{                           knn\_metrics}
\NormalTok{                      ) }\SpecialCharTok{\%\textgreater{}\%} 
  \CommentTok{\# get rid of \textquotesingle{}sec elapsed\textquotesingle{} and turn it into a number}
  \FunctionTok{mutate}\NormalTok{(}\AttributeTok{time =} \FunctionTok{str\_sub}\NormalTok{(time, }\AttributeTok{end =} \SpecialCharTok{{-}}\DecValTok{13}\NormalTok{) }\SpecialCharTok{\%\textgreater{}\%} 
           \FunctionTok{as.double}\NormalTok{()}
\NormalTok{         )}
\end{Highlighting}
\end{Shaded}

\hypertarget{calculating-the-cost-of-fraud-to-the-company}{%
\subsection{Calculating the cost of fraud to the
company}\label{calculating-the-cost-of-fraud-to-the-company}}

\begin{itemize}
\tightlist
\item
  How much money (in US\$ terms) are fraudulent transactions costing the
  company? Generate a table that summarizes the total amount of
  legitimate and fraudulent transactions per year and calculate the \%
  of fraudulent transactions, in US\$ terms. Compare your model vs the
  naive classification that we do not have any fraudulent transactions.
\end{itemize}

\begin{Shaded}
\begin{Highlighting}[]
\NormalTok{best\_model\_preds }\OtherTok{\textless{}{-}} 
\NormalTok{  rf\_wflow }\SpecialCharTok{\%\textgreater{}\%} 
  \FunctionTok{fit}\NormalTok{(}\AttributeTok{data =}\NormalTok{ card\_fraud\_train) }\SpecialCharTok{\%\textgreater{}\%}  
  
  \DocumentationTok{\#\# Use \textasciigrave{}augment()\textasciigrave{} to get predictions for entire data set}
  \FunctionTok{augment}\NormalTok{(}\AttributeTok{new\_data =}\NormalTok{ card\_fraud)}

\NormalTok{best\_model\_preds }\SpecialCharTok{\%\textgreater{}\%} 
  \FunctionTok{conf\_mat}\NormalTok{(}\AttributeTok{truth =}\NormalTok{ is\_fraud, }\AttributeTok{estimate =}\NormalTok{ .pred\_class)}
\end{Highlighting}
\end{Shaded}

\begin{verbatim}
##           Truth
## Prediction      1      0
##          1   2212    116
##          0   1724 666976
\end{verbatim}

\begin{Shaded}
\begin{Highlighting}[]
\NormalTok{cost }\OtherTok{\textless{}{-}}\NormalTok{ best\_model\_preds }\SpecialCharTok{\%\textgreater{}\%}
  \FunctionTok{select}\NormalTok{(is\_fraud, amt, }\AttributeTok{pred =}\NormalTok{ .pred\_class) }

\NormalTok{cost }\OtherTok{\textless{}{-}}\NormalTok{ cost }\SpecialCharTok{\%\textgreater{}\%}
  \FunctionTok{mutate}\NormalTok{(naive\_false }\OtherTok{\textless{}{-}} \FunctionTok{sum}\NormalTok{(pred }\SpecialCharTok{!=}\NormalTok{ is\_fraud),}
\NormalTok{false\_negatives }\OtherTok{\textless{}{-}} \FunctionTok{sum}\NormalTok{(pred }\SpecialCharTok{==} \DecValTok{0} \SpecialCharTok{\&}\NormalTok{ is\_fraud }\SpecialCharTok{==} \DecValTok{1}\NormalTok{),}
\NormalTok{false\_positives }\OtherTok{\textless{}{-}} \FunctionTok{sum}\NormalTok{(pred }\SpecialCharTok{==} \DecValTok{1} \SpecialCharTok{\&}\NormalTok{ is\_fraud }\SpecialCharTok{==} \DecValTok{0}\NormalTok{),}
\NormalTok{true\_positives }\OtherTok{\textless{}{-}} \FunctionTok{sum}\NormalTok{(pred }\SpecialCharTok{==} \DecValTok{1} \SpecialCharTok{\&}\NormalTok{ is\_fraud }\SpecialCharTok{==} \DecValTok{1}\NormalTok{),}
\NormalTok{true\_negatives }\OtherTok{\textless{}{-}} \FunctionTok{sum}\NormalTok{(pred }\SpecialCharTok{==}\DecValTok{0}\SpecialCharTok{\&}\NormalTok{is\_fraud}\SpecialCharTok{==}\DecValTok{0}\NormalTok{))}
  

  \CommentTok{\# naive false{-}{-} we think every single transaction is ok and not fraud}


  \CommentTok{\# false negatives{-}{-} we thought they were not fraud, but they were}

  
  
  \CommentTok{\# false positives{-}{-} we thought they were fraud, but they were not}

  
    
  \CommentTok{\# true positives{-}{-} we thought they were fraud, and they were }


  
  \CommentTok{\# true negatives{-}{-} we thought they were ok, and they were }

  
\CommentTok{\# Summarising}

\NormalTok{cost\_summary }\OtherTok{\textless{}{-}}\NormalTok{ cost }\SpecialCharTok{\%\textgreater{}\%} 
  \FunctionTok{summarise}\NormalTok{(}\FunctionTok{across}\NormalTok{(}\FunctionTok{starts\_with}\NormalTok{(}\FunctionTok{c}\NormalTok{(}\StringTok{"false"}\NormalTok{,}\StringTok{"true"}\NormalTok{, }\StringTok{"amt"}\NormalTok{)), }
            \SpecialCharTok{\textasciitilde{}} \FunctionTok{sum}\NormalTok{(.x, }\AttributeTok{na.rm =} \ConstantTok{TRUE}\NormalTok{)))}

\NormalTok{cost\_summary}
\end{Highlighting}
\end{Shaded}

\begin{verbatim}
## # A tibble: 1 x 5
##   false_negatives <- sum(pred ==~1 false_positives <- s~2 true_positives <- su~3
##                              <int>                  <int>                  <int>
## 1                       1156852272               77839248             1484313936
## # i abbreviated names: 1: `false_negatives <- sum(pred == 0 & is_fraud == 1)`,
## #   2: `false_positives <- sum(pred == 1 & is_fraud == 0)`,
## #   3: `true_positives <- sum(pred == 1 & is_fraud == 1)`
## # i 2 more variables: `true_negatives <- sum(pred == 0 & is_fraud == 0)` <dbl>,
## #   amt <dbl>
\end{verbatim}

\begin{Shaded}
\begin{Highlighting}[]
\FunctionTok{glimpse}\NormalTok{(cost\_summary)}
\end{Highlighting}
\end{Shaded}

\begin{verbatim}
## Rows: 1
## Columns: 5
## $ `false_negatives <- sum(pred == 0 & is_fraud == 1)` <int> 1156852272
## $ `false_positives <- sum(pred == 1 & is_fraud == 0)` <int> 77839248
## $ `true_positives <- sum(pred == 1 & is_fraud == 1)`  <int> 1484313936
## $ `true_negatives <- sum(pred == 0 & is_fraud == 0)`  <dbl> 447559571328
## $ amt                                                 <dbl> 47183904
\end{verbatim}

\begin{itemize}
\item
  If we use a naive classifier thinking that all transactions are
  legitimate and not fraudulent, the cost to the company is .
\item
  With our best model, the total cost of false negatives, namely
  transactions our classifier thinks are legitimate but which turned out
  to be fraud, is .
\item
  Our classifier also has some false positives, , namely flagging
  transactions as fraudulent, but which were legitimate. Assuming the
  card company makes around 2\% for each transaction (source:
  \url{https://startups.co.uk/payment-processing/credit-card-processing-fees/}),
  the amount of money lost due to these false positives is
\item
  The \$ improvement over the naive policy is .
\end{itemize}

\end{document}
